\documentclass[addpoints]{exam}

\newenvironment{palatino}{\fontfamily{ppl}\selectfont}{\par}
\usepackage[english]{babel}
\usepackage[utf8]{inputenc}
\usepackage[T1]{fontenc}
\usepackage{graphicx}% Include figure files
\usepackage{comment}
\usepackage{multicol}
\usepackage{placeins}
\usepackage{amsmath,amssymb,amsthm,mathrsfs,amsfonts,dsfont} 
\usepackage{hyperref}
%
\addtolength{\topmargin}{-1cm}
	\addtolength{\textheight}{2cm}
	\pagestyle{empty}
\begin{document}
\begin{palatino}

%\maketitle
%\pagenumbering{gobble}
\noindent Corso di python novembre-dicembre 2020 - ISM \\
\noindent Fulvio Paleari -- fulvio.paleari@mlib.ism.cnr

\hrulefill
%\centering
\vspace{0.2in}\\
\setlength{\parindent}{0pt}

\textbf{1. Introduzione a python I} \\

a. \quad Utilizzando \texttt{numpy}, generate una matrice $3\times 3$ contentente elementi casuali compresi fra $-1$ e $1$ e trovatene il determinante utilizzando la regola di Sarruss. Fate una funzione per generare la matrice e un'altra per il calcolo del determinante. Controllate il risultato utilizzando il metodo di \texttt{numpy} per trovare il determinante di una matrice.

\hphantom{ciao}

b. Considerando la matrice $3\times 3$ costruita precedentemente, calcolatene autovalori e autovettori con gli opportuni metodi \texttt{numpy}, dopodiché visualizzatene lo spettro usando \texttt{matplotlib.pyplot} secondo la seguente formula:
\begin{equation*}
	S(\omega)= \mathrm{Im}\left\{\sum_{i=1}^3 \frac{|\mathbf{e}_i|^2}{\omega-E_i-\mathrm{i}\eta} \right\},
\end{equation*} 
dove $\mathbf{e}_i$ è un autovettore e $E_i$ il corrispondente autovalore della matrice $3\times 3$, mentre $\eta>0$ rappresenta un parametro di \textit{linewidth} (molto piccolo rispetto alla scala di $\omega$ considerata) introdotto \textit{ad hoc}.

\hphantom{ciao}

c. \quad La sequenza di Fibonacci è definita dalla seguente realazione ricorsiva:

\begin{equation*}
	F_n = F_{n-1}+F_{n-2} \qquad \mathrm{con} \ F_1 = 1 \ e \ F_2 = 1.
\end{equation*}

Quindi i primi $12$ termini saranno:

\begin{equation*}
	F_1 = 1, \
	F_2 = 1, \
	F_3 = 2, \
	F_4 = 3, \
	F_5 = 5, \
	F_6 = 8, \
	F_7 = 13, \
	F_8 = 21, \
	F_9 = 34, \
	F_{10} = 55, \
	F_{11} = 89, \
	F_{12} = 144.
\end{equation*}

Il dodicesimo termine, $F_{12}$, è il primo termine formato da tre cifre.

Qual è l'indice del primo termine della sequenza di Fibonacci formato da $10$ cifre? 
Potete usare funzioni, cicli \texttt{for}, \texttt{if} e \texttt{numpy} array.

\hphantom{ciao}

[Questa è una versione del problema $25$ di \href{https://projecteuler.net/problem=25}{Project Euler}]

\hphantom{ciao}
 
d. \quad 	

\hphantom{ciao}
	
\textbf{3. Visualizzazione dati} \\

\hphantom{ciao}


\textbf{4. Analisi dati} \\


\hphantom{ciao}

\textbf{5. Programmazione orientata ad oggetti}

\hphantom{ciao}

a. \quad È possibile stimare il valore di $\pi$ attraverso una simulazione MonteCarlo.
Per farlo si campiona l'area di un quadrato di lato uguale a 2, ovvero si estraggono attraverso un generatore di numeri random coppie $(x,y)$ di numeri nell'intervallo $[-1,1]$.
Per ogni coppia estratta si controlla se il punto è interno a un cerchio di raggio $1$, ovvero si verifica che $x^2+y^2 <= 1$. 
Il rapporto fra il numero di punti interni al cerchio e il numero totale di punti tende al valore $\pi/4$ al crescere del numero di campionamenti.

\hphantom{ciao}

Realizzate una classe python che gestisca questa simulazione. 
La classe richiede il numero di campionamenti come parametro da passare al costruttore (metodo \texttt{\_\_init\_\_}) e definisce un metodo \texttt{run} che esegue la simulazione. 
Oltre al valore finale il metodo deve produrre il plot di convergenza, ovvero il plot del valore del rapporto durante la simulazione. 
A questo scopo è necessario definire come membro della classe una lista, o un array \texttt{numpy}, che contenga i valori del rapporto punti interni / punti totali durante la simulazione.

\hphantom{ciao}

[Esercizio proposto da Marco D'Alessandro]

\FloatBarrier

\end{palatino}
\end{document}